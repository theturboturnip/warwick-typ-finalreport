\subsection{Background}
One of the first papers on optimizing a CFD simulation was released in 1995\cite{paper:1995CfdOpt:1383209}.
This paper considered the effect of automatic compiler parallelization and optimization of a full CFD program, and the steps a programmer must take to guide the compiler i.e. avoiding false sharing.
The program was only executed on the CPU, as General Purpose GPU computing (GPGPU) had not yet taken hold.

GPGPU was first used for CFD simulations in 2004 with this paper\cite{paper:2004CfdGPU:10.1109/SC.2004.26}.
This used the ``fragment shading'' stage of the GPU rendering pipeline to perform the computation, as standalone ``compute'' pipelines were only exposed by APIs from 2007 onwards.
Such APIs include CUDA (2007)\cite{tool:CUDAProgrammingV1}, OpenCL (2008)\cite{tool:OpenCL1.0PressRelease}, DirectX's DirectCompute (2009)\cite{tool:DirectComputePresentation}, and OpenGL 4's compute shaders (2012)\cite{tool:OpenGLComputeShaderExt}.
% TODO - OpenGL compute extensions may have existed before this.

Since 2007, using GPGPU for CFD has become a large topic of study, as investigated in detail by \cite{paper:GPGPUSummary:10.1007/s11227-013-1015-7}.
We will take advantage of this to optimize the simulation speed to the point that a simulation can be both performed and visualized in real time, which many other programs do not achieve.
% While the concept of accelerating a fluid simulation on the GPU is not new, much of the novelty of our optimizations will stem from the 
% interaction between the simulation and the other systems at work.
% As an example, if the simulation were to be modifiable with user input, introducing this new data and updating the boundary conditions in an efficient manner becomes a new problem.
% \todomark{Interactivity not a thing lmao}