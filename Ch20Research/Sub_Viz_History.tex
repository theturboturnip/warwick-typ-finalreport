\subsection{Background}
%Earliest found instance of interactive visualizaion is (https://dl.acm.org/doi/10.5555/509740.509745) in 2002. Did not use the GPU for any computation.
One of the earliest CFD interactive visualizations was in 2002, which had a simulation running slower than real time on a separate computer to the real-time visualization\cite{paper:2002vis:10.5555/509740.509745}.
%\todocite{https://dl.acm.org/doi/10.5555/509740.509745}
Decoupling the simulation speed from the visualization speed allowed for high framerates to be achieved for the user interface, however any changes made from the user interface had a delay of 0.5 seconds before being reflected in the simulation.

Many scientific visualizations of fluid flow exist already.
% To name two examples, streak lines are commonly used for visualizing velocities in a still image\todocite{}, and colored smoke is used to visualize airflow on moving objects in videos\todocite{}.
To name two examples, streak lines and fluid colors are used for visualizing fluid flow\cite{video:AutodeskFlowDesign}.
These methods are perfectly fine for those who understand what these elements mean, i.e. what the colors represent, and what the optimal airflow would look like.
However, for those unfamiliar with the simulation these methods can be difficult to understand.

This project aims to develop new visualization techniques for two-dimensional simulations that are more intuitive than the current offerings, that can be extended to three dimensions easily.
Using high-speed rendering APIs like Vulkan\cite{tool:Vulkan} will allow these visualizations to be made even more complex while maintaining high speeds.
Furthermore, our approach may allow for slight inaccuracies to be introduced for the sake of intuitivity, which has not been explored in research to the author's knowledge.

% However novelty in the visualization space not pursued.

% Introduction of Vulkan allows for more efficient rendering, which can be made more complex than before.