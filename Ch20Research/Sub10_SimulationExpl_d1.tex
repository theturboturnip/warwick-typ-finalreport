\newcommand{\deltaT}[0]{$\delta{}t$}
\newcommand{\deltaX}[0]{$\delta{}x$}
\newcommand{\deltaY}[0]{$\delta{}y$}

\section{An Example Simulation Tick}
% Cite book(cite) as inspiration for CFD code
The 1998 book ``Numerical simulation in fluid dynamics : a practical introduction''\cite{book:griebel1998numerical} defines a basic structure for a discrete simulated timestep (a.k.a. a ``tick'') and provides a sample guide to implementing it in Fortran or C.
To the best of the author's knowledge this was used as the base of the ACA coursework, and continues to be the base of this project.
This section will explain the general structure of the simulation as defined in \cite{book:griebel1998numerical}.

% Note the kind of simulation - incompressible laminar flow?
% Also define Reynolds number, it comes up in places
% The simulation is intended to simulate ``laminar flow of viscous, incompressible fluids''\cite{book:griebel1998numerical}.
% This has a few implications:
% \begin{itemize}
%     \item The fluid is viscous, which means the internal friction forces are non-negligible. This is as opposed to a gaseous fluid which has n
%     \item The flow is \textit{laminar}, not \textif{turbulent}, which means 
% \end{itemize}

% 1/2 paragraphs on the grid structure
The simulation (when in 2D) solves for three variables: horizontal velocity, vertical velocity, and pressure.
These variables are represented by $u$, $v$, and $p$ respectively.
At timestep $\#n$, they may be shown as $u^{(n)}, v^{(n)}, p^{(n)}$.
To discretize these values, they are evaluated at points on a staggered grid\todocite{Figure}.
\todomark{explain how the variables are staggered}
Staggering the variables in this manner avoids oscillation caused by odd-even decoupling\todocite{https://www.cfd-online.com/Wiki/Staggered_grid, Harlow, F. H. and Welch, J. E. (1965), "Numerical calculation of time-dependent viscous incompressible flow of fluid with free surface", Phys. Fluids. 8, 2182. }.
\todomark{This could also be solved by colocated grids?}\todocite{}
It is important to note the variables $u,v$ represent the current velocity of the fluid within each grid space, \textbf{not} the velocity of the grid cells themselves.
The grid does not move at any point during the simulation.

The grids are indexed by $i$ in the x-direction and $j$ in the y-direction.
To allow for derivatives to be accurately calculated for cells on the edges of the grid, boundary cells are added around each grid.\todomark{TODO figure}
For a finite domain of size $(imax, jmax)$ this leads to a final grid size of $(imax + 1)$ by $(jmax + 1)$, where valid values fall in the ranges %
%$1 \leq i \leq imax$, $1 \leq j \leq jmax$
$i \in \{1..imax\}$, $j \in \{1..jmax\}$.
The physical dimensions of each grid space are represented by \deltaX{}, \deltaY{}.

The derivatives of $u$ and $v$ can then be calculated by finding the centered differences\todocite{}.
\begin{equation}
    \left[\frac{\partial{u}}{\partial{x}}\right]_{i,j} := \frac{u_{i,j}-u_{i-1,j}}{\delta{x}},%
    \left[\frac{\partial{v}}{\partial{y}}\right]_{i,j} := \frac{v_{i,j}-v_{i,j-1}}{\delta{y}}
\end{equation}
% TODO multicolumn
% \begin{multicols}{2}
%   \begin{equation}
%     a=b
%   \end{equation}\break
%   \begin{equation}
%     b=c
%   \end{equation}
% \end{multicols}
The partial derivatives for pressure $\partial{p}/\partial{x}, \partial{p}/\partial{y}$ are found in the same way.
The remaining derivatives, including second derivatives and $\partial{uv}/\partial{x}, \partial{uv}/\partial{y}$, can also be discretized by taking the difference across midpoints of their respective dimensions (see \todocite{book or [Hirt et al. 1975]}).
%

\subsection{Timestep Calculation}
% TODO read up more and check back lol
As the derivatives are calculated between adjacent grid points, it is impossible to accurately simulate a timestep where fluid moves between non-adjacent grid cells \todomark{Figure?}.%[Peyret&Taylor,1983]or[Roache,1976].Anadaptivestepsizecontrolbasedonthesestabilityconditionsis usedin[Tome&McKee,1994].
% The simulation cannot accurately solve a timestep in which particles move completely over a cell (see Figure \todocite{}).
To prevent this issue the timestep \deltaT{} is calculated from the fluid velocities to make this impossible.
\begin{equation}
    \delta{t} = \tau min\left(
        \frac{Re}{2}\left(
            \frac{1}{\delta{x}^2} + \frac{1}{\delta{y}^2}
        \right)^-1,
        \frac{\delta{x}}{|u_max|},
        \frac{\delta{y}}{|v_max|}
    \right)
\end{equation}

Because the new velocities may increase from $u_max$ and $v_max$, the safety factor $\tau \in [0, 1]$ is used to ensure the timestep is large enough to account for it.\todomark{Is this right?}

\subsection{Tentative Velocity}
The final values of $u$ and $v$ are defined as \todomark{COMPLETE EQ}
\begin{equation}
\begin{aligned}
    u^{(n+1)} &= u^{(n)} + \delta{t}\left[\frac{1}{Re}\left(\right)\right] ETC (see page 48). \\
    v^{(n+1)} &= v^{(n)} + etc.
\end{aligned}
\end{equation}
However, as these depend on the partial derivatives of $p^{(n+1)}$, which itself depends on velocity, they cannot be solved analytically.
In order to iteratively find $p$ the variables $f$ and $g$, for horizontal and vertical ``tentative velocity'', are introduced.\todomark{DEFINE f,g}
\begin{equation}
\begin{aligned}
    f^{(n)} := etc \\
    g^{(n)} := etc
\end{aligned}
\end{equation}
\begin{equation}
\begin{aligned}
    u^{(n+1)} = f^{(n)} - \delta{t}\frac{\partial{p^{(n+1)}}}{\partial{x}} \\
    v^{(n+1)} = g^{(n)} - \delta{t}\frac{\partial{p^{(n+1)}}}{\partial{y}}
    \label{eq:uv_modified}
\end{aligned}
\end{equation}

\subsection{Solving the Poisson Equation with SOR}
% Two phases - calculating the RHS, then solving
For continuity to be achieved, the final velocity values must fulfil
\begin{equation}
    \frac{\partial{u^{(n+1)}}}{\partial{x}} + \frac{\partial{v^{(n+1)}}}{\partial{y}} = 0
\end{equation}
This means that the total amount of fluid entering a cell in tick $n+1$ is equal to the amount of fluid leaving, which must be the case otherwise the amount of fluid per cell wouldn't be constant and the fluid would be \textbf{compressed}.

Substituting the formulae in \autoref{eq:uv_modified} into this relation and rearranging gives
\begin{equation}
    \frac{\partial^2{p^{(n+1)}}}{\partial{x^2}} + \frac{\partial^2{p^{(n+1)}}}{\partial{y^2}} = \frac{1}{\delta{t}}\left(\frac{\partial{f^{(n)}}}{\partial{x}} + \frac{\partial{g^{(n)}}}{\partial{y}}\right)
\end{equation}
The right hand side of this equation is constant for timestep $n$, so can be precalculated and assigned to its own variable $rhs$
\begin{equation}
rhs := \frac{1}{\delta{t}}\left(\frac{\partial{f^{(n)}}}{\partial{x}} + \frac{\partial{g^{(n)}}}{\partial{y}}\right)
\end{equation}
\begin{equation}
\frac{\partial^2{p^{(n+1)}}}{\partial{x^2}} + \frac{\partial^2{p^{(n+1)}}}{\partial{y^2}} = rhs
\end{equation}

Once rhs is found the remaining second-order differential equation can be solved using Red/Black Successive Over-Relaxation SOR.
In this method, the grid is split into red and black squares with a checkerboard pattern\todomark{figure}.
First the values at red squares are updated, which depend only on the black squares.
Then, the black squares are updated, which depend only on the red squares.
\todomark{An explanation of what SOR does, how it works, how residual calculations work.}

\subsection{Final Velocity Calculations}
% Update Velocity, applyBoundaryConditions
Once the final values of $p$ have been calculated the velocity values $u,v$ can be found with \autoref{eq:uv_modified}.
The boundary conditions for velocity must also be applied \todomark{DESCRIBE THEM}.


% TODO - tick pipeline figure
% TODO - find like 5 citations here at least