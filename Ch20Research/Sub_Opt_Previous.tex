\subsection{Previous Work}
\label{research:prev_work}
As work on CFD progressed some optimizations were developed that change the simulation pipeline and provide an overall speedup.
Some of these were adapted into my ACA coursework submission\cite{modules:aca257submission}, which this project is based on, and carry over into the CUDA version.


Given the definition of $\beta$ in \cref{eq:poisson_beta}, the value of $\beta_{i,j}$ does not change over the course of the simulation and so can be precalculated before the simulation starts.
Additionally if it can be guaranteed that for every boundary square $p = 0$, which can be done either by never updating their pressure values or by updating them with $\beta_{i,j} = 0$, then $\epsilon_{i,j}$ doesn't need to be evaluated during the simulation at all.
These optimizations increased the runtime speed of the Poisson evaluation by 2.24x\footnote{The $\beta$ precalculation increased speed by 1.4x, and the removal of $\epsilon$ increased speed by 1.6x.\cite{modules:aca257submission}}, and they have been kept in the CUDA program.

The book states an alternate solution where $\epsilon$ is set to 1 at all times and pressure values on boundaries are copied from adjacent fluid squares\cite{book:griebel1998numerical}.
This apparently stops noncontinuous starting velocities from producing nonphysical pressure values.
\label{ext:PressureValues}

% 1 on red/black
% The example implementation and the ACA implementation both use Red-Black Successive Over-Relaxation (SOR).
% Red-black SOR was first used to solve a system of linear equations on vector and parallel computersystems by Adams and Ortega [1], although introduced earlier (e.g., by Young [119]). Liu et al. [65] provideda more detailed analysis of red-black SOR implemented on GPUs
% NOTE - the Liu et al. there is a pretty close project

% In this method, the grid is split into red and black squares with a checkerboard pattern\todomark{figure}.
% First the values at red squares are updated, which depend only on the black squares.
% Then, the black squares are updated, which depend only on the red squares.
As stated in \cref{sec:SimulationPoisson}, red/black SOR is used to iteratively solve the Poisson equation.
% The red and black operations can be performed in parallel, but in both the ACA coursework and the CUDA program this has not been done\footnote{It has also been marked as a potential extension}.
In the initial ACA coursework the values ($f$, $g$, $p$, $rhs$) for red and black data were stored in the same arrays.
This was problematic as data of the same color was never contiguous, and any iteration looking for just red values would get a cache line with both colors, leading to half of each cache line being wasted.
To fix this, red and black data is split into separate arrays before starting the Poisson solver.
This has been carried over into the CUDA implementation.

% Parallelization i.e. Vectorization, OpenMP?
% "The ACA coursework used OpenMP to provide thread-level parallelization in kernels, which is automatically provided when using a GPU."
The ACA solution used OpenMP\cite{OpenMPHomeOpenMP} to automatically parallelize the Poisson solver (and other program elements) by column.
That is, each thread was given a group of columns to process.
This was not needed in the CUDA version as each GPU kernel is implicitly parallelized over many GPU threads.

The ACA solution included optimizations exploiting properties of the original code, such as floating point precision, to speed up calculations while producing identical results.
These optimizations include using fused multiply-add\cite{Muller2010TheInstruction} in some places (but not all), precalculating divisions with double-precision floats, and skipping the residual calculation phase altogether.
As this project is focused on improving upon the accuracy of the ACA submission, instead of producing bit-identical results, these optimizations have generally not been implemented into the CUDA program.\footnote{The CUDA program still lacks a residual phase, but this is planned to be implemented later.\label{sec:OptimizationReAddingResidual}}

% Paragraph on ACA-specific (FMA tricks, removing divisions, residual removal)
% "Other courseworks specifically tied to the ACA coursework were"...

%% ACA Recap(cite)
%% - Residual removal. (probably needs to be reversed)
%% - beta precalc (check if somewhere/ in book)
%% - mathematical optimization to remove branches (check if in book)
%% - FMA trickery (cite FMA properties)
%% - red/black rearranging (defo mentioned in top level paper, find citation from that)
%% - vectorization AVX/SSE (can most likely cite something here)
%%     - Note that 8-vector was tested,but 4-vector was faster
%% - removing divisions (maybe?)
%% - OpenMP (irrelevant here but can provide a citation? and should be noted because it provides parallelism which is then replaced by the GPU)


% TODO - tick pipeline figure with ACA additions (ex. r/b splitting)
