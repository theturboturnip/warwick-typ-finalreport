\subsection{Future Work}
% Boids(cite), particles(cite), things proposed by the book
While a purely image-based approach to visualizing properties can be useful, other approaches allow for i.e. multidimensional quantities such as velocity to be expressed much more easily.
In the case of velocity, vector fields and particle tracing are both shown in \cite{book:griebel1998numerical} to be effective.

Given that our simulation is realtime, we can also add changes over time to the mix.
Tracing particle paths and rendering them as a line could be replaced by actually watching the particles move over time.
Particle movement could also be enhanced with extra behaviour similar to that of BOIDs\cite{BOIDS_10.1145/37401.37406}, which among other things implement Collision Avoidance.
This would prevent particles from overlapping and getting visually lost.
Vorticity/rotational movement could be visualized by adding particles to the grid that rotate over time based on he vorticity at their location.
This could allow the vorticity to be represented in the same view as the other parts of the simulation, instead of creating a dedicated view separate from velocity/pressure.