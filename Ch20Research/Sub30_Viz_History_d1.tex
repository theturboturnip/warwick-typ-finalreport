\subsection{Background}

% \todomark{New viz research background ya dingus}

% Para on normal viz, explain that it's old as a concept

% Introduce in-situ visualization

One of the earliest CFD interactive visualizations was in 2002, which had a simulation running slower than real time on a separate computer to the real-time visualization\cite{paper:2002vis:10.5555/509740.509745}.
%\todocite{https://dl.acm.org/doi/10.5555/509740.509745}
Decoupling the simulation speed from the visualization speed allowed for high framerates to be achieved for the user interface, however any changes made from the user interface had a delay of 0.5 seconds before being reflected in the simulation.
This qualifies as a loosely-coupled in-situ visualization, because simulation data is streamed to a separate visualization system while the simulation completes.

% Introduce current industrial tools (autodesk CFD, VTK)
A significant open-source toolkit is VTK\cite{VTKWebpage}, first introduced in 1993\cite{VTKBook}, which powers multiple tools such as ParaView\cite{ParaViewWebpage} and VisIt\cite{VisItWebpage}.
Both of these programs support tightly-coupled in-situ visualization of an external simulation with plugins, but the VTK base is not well suited to in-situ integration\cite{kress2017situ}.

Autodesk CFD\cite{AutodeskCFDWebpage} is a closed-source tool which integrates its own simulation and visualization together.
It's based on ALGOR FEA, created by ALGOR Inc which was acquired by Autodesk in 2008 \cite{AutodeskAcquiresALGOR}.
The latest iteration is targeted at the manufacturing industry, unlike VTK which is a generic toolkit, and does not support in-situ visualization.

Both of the above examples are not built for in-situ visualization, but these programs still represent the industry standard for visualization capabilities.
A novel element of this project is building a program from the ground up for in-situ visualization, and the visualization components will be based on Autodesk.