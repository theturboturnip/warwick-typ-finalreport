\subsection{Library Selection}
\label{sec:LibrarySelection}
\begin{figure}[ht]
    \centering
    \begin{tabular}{r|c c}
        & OpenGL & Vulkan \\
        \hline
        OpenCL & Y & N \\ 
        CUDA & Y & Y \\ 
        OpenGL & Y & N \\ 
        Vulkan & N & Y \\ 
    \end{tabular}
    \caption{Graphics and Compute Backend Interoperability Matrix}
    \label{fig:LibraryChoices}
\end{figure}
CUDA and Vulkan have been chosen as backends, but other backends were also considered.
As the simulation would have to run on DCS systems (\cref{reqN:DCSCompile}) and thus run on Linux, the only possible GPU rendering backends were OpenGL and Vulkan.
However, there were still multiple choices of compute backend:
\begin{itemize}
    \item OpenCL\cite{tool:OpenCL1.0PressRelease} is an ``Open Standard for Parallel Programming of Heterogeneous Systems''\cite{TheKhronosGroupOpenCLInc}.
    \item CUDA\cite{tool:CUDA} is a closed-source library for running parallel code on NVIDIA GPUs.
    \item OpenGL has Compute Shaders\cite{tool:OpenGLComputeShaderExt} which can execute computations outside of the graphics pipeline.
    \item Vulkan also has Compute capability\cite{TheKhronosGroupVulkanGuide}, similar in function to OpenGL.
\end{itemize}
To decide on the compute backend to use, an interoperability matrix was drawn (\cref{fig:LibraryChoices}) to show which libraries could share data without copying it between buffers.
As the researcher was already experienced with Vulkan, and the more granular control it provides would be beneficial to performance, Vulkan was selected as the rendering backend.
This prevented OpenCL and OpenGL from being used as compute backends, as they are not compatible with Vulkan.
CUDA and Vulkan have comparable ability, but CUDA was chosen as the compute backend.
The Vulkan compute shaders are still a very graphics-oriented view of computation, and CUDA would give the researcher experience with other kinds of libraries.
A Vulkan compute backend is used for the visualization portion of the code.

% Not for now, but note that the C++ vulkan bindings are nice. They do end up wrapped in other classes, but they remove the need to remember boilerplate as much.

In other cases, there were clear choices: the SDL2\cite{SimpleHomepage} window and input library and the Dear ImGUI\cite{CornutDearImGui} UI library were chosen due to personal experience.
The \shell{stb\_image.h} header was found to be a simple method of importing image colour data as byte arrays, used for the input generator (\cref{req:GenerateState}).

There are a great many options for Command-Line parsing libraries, even more so because C++ is used instead of C.
A recent survey of the possibilities\cite{attractivechaos2018AC/C++} was whittled down to five options.

\code{getopt}\cite{FreeSoftwareFoundationGetopt3:Page}, \code{argp}\cite{GNUProjectArgpLibrary}, and \code{gopt}\cite{VajzovicGoptLibrary} are C libraries that use arrays of structures to define the required arguments.
Of them, only \code{argp} can automatically generate a \shell{--help} argument, which is a very valuable feature.
\code{cxxopts}\cite{jarro2783Cxxopts:Parser} was considered as a C++ alternative but used very odd syntax for defining arguments.
Ultimately CLI11\cite{CLIUtilsCLI11} was chosen as a modern C++11 library that had native support for subcommands, which were used heavily for separating program components (see \cref{sec:DesignSubcommands}).