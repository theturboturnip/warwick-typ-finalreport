\section{Code Safety}\label{sec:Impl:CodeSafety}
No program is faultless, and when faults manifest during execution it's important to ensure they have a minimal effect on their surroundings.
This program includes many means of detecting errors both at compile-time and run-time to ensure its dependability.

The G++ flag \shell{-Wall} enables many warnings that are emitted at compile-time if potential errors are detected.
Unlike some other languages (i.e. Verilog) these warnings are generally reliable and it is feasible to build a C++ program that compiles without any warnings.
To ensure this the \shell{-Werror} flag is added to upgrade these warnings to compilation errors, ensuring a program which compiles will be warning-free\footnote{Some warnings, such as those from \shell{-Wextra} and those relating to unused variables and parameters, are usually benign so weren't upgraded.}.

Runtime errors are detected with a set of C++ macros that check if required conditions are met.
As there is no liveness requirement for the program, failure triggers an immediate program exit to avoid errors propagating through the system.
The \code{DASSERT} family of macros are included only in Debug builds, and the \code{FATAL\_ERROR} family of macros test both in Release and Debug.
Both families print a message including the file and line of code that triggered the error, and any other relevant debug information.
These families are also integrated with the \code{CHECKED\_CUDA} and \code{CHECKED\_VULKAN} macro families, which surround CUDA/Vulkan function calls and check the returned error codes.
An example is shown in \cref{fig:ImplAssertions}.
\begin{figure}[t]
    \centering
    \begin{subfigure}{0.49\textwidth}
        \begin{minted}{cpp}
if (value == unexpected) {
    FATAL_ERROR(
        "Unexpected Value %d\n",
        value
    );
}
// equivalent to
FATAL_ERROR_IF(
    value == unexpected,
    ...
);
// or
FATAL_ERROR_UNLESS(
    value != unexpected,
    ...
);

// Same, but only fails in Debug builds
DASSERT(value != unexpected);
        \end{minted}
        \caption{Example of assertion macros}
    \end{subfigure}%
    \begin{subfigure}{0.49\textwidth}
        \begin{minted}{cpp}
cudaError_t error = cudaDeviceSynchronize();
FATAL_ERROR_IF(error != cudaSuccess);
// Equivalent to
CHECKED_CUDA(cudaDeviceSynchronize());

auto result = vkDeviceWaitIdle();
FATAL_ERROR_IF(result != vk::Result::eSuccess);
// Equivalent to
CHECKED_VULKAN(vkDeviceWaitIdle());
        \end{minted}
        \caption{Example of API failure safety macros}
    \end{subfigure}
    \caption{Examples of error safety via macros}
    \label{fig:ImplAssertions}
\end{figure}

The final tool for error detection is the Vulkan Validation Layer.
This is a standard Vulkan extension which checks each Vulkan call to ensure the Vulkan specification \cite{TheKhronosGroupVulkanSpec} isn't violated.
These checks are very in-depth, and are a must-have when debugging visualization errors.
The base Vulkan functions don't do this error checking for efficiency's sake, and the program disables these layers in Release mode for the same reason.

\subsection{Smart Resource Classes}
All memory allocations, CUDA objects, and Vulkan objects follow the same allocate/release pattern.
They are not destroyed automatically when they go out of scope, but must be released manually.
This is an error-prone process, as a programmer may forget which resources need to be released or try to release a resource twice.

\begin{figure}
    \centering
    \begin{cppcode}
// Manual handling
{
    auto* semaphore = new Semaphore();
    // do things with semaphore...
    delete semaphore;
}

// Automatic handling
{
    std::unique_ptr<Semaphore> semaphore = std::make_unique<Semaphore>();
    // do things with semaphore...
    // automatically deleted
}
    \end{cppcode}
    \caption{Example of memory management with C++ standard classes}
    \label{fig:ImplUniquePtr}
\end{figure}

Smart resource classes alleviate this by tying the resource lifetime to a C++ object, using the object \emph{destructor} to release the resource.
The prime example of this is \code{std::unique_ptr<T>}, which holds a \code{T*} object and \code{free()}-s it when the object leaves scope (\cref{fig:ImplUniquePtr}).
However this does not easily map to other deletion methods, and the pointer adds an unwanted extra level of indirection.
\code{vulkan.hpp}, included in the Vulkan SDK, includes similar classes for each type of Vulkan resource, but still requires a lot of boilerplate to initially create objects\footnote{Vulkan objects require a full CreateInfo struct to create, rather than taking function arguments.}.
In some cases, it's also more convenient to store multiple resources together, such as a buffer and the device memory it uses, which cannot be directly achieved with either method.
To solve these problems custom smart resource classes are implemented for individual resources and aggregates, with more convenient constructors (see \cref{sec:appx_resourceclasses}).

These smart classes are affected by C++ copy/move semantics.
C++ allows objects to be copied with a copy-constructor, or moved with a move-constructor.
The resource objects should not be copyable as it would become unclear which copy would be responsible for destroying the resource.
The copy-constructor can be deleted to prevent this, but the move-constructor is useful for transferring ownership e.g. from a resource factory to the person using the resource.
When an object is moved, the original version should forget the data it's holding and give it to the new object.
C++ can generate this constructor automatically, but this ``default move-constructor'' will just copy the data across if the data is \textit{TriviallyCopyable}\cite{cpprefMoveConstructor}.%, and is true for pointers and all CUDA handles.
All pointers and CUDA handles are \textit{TriviallyCopyable}, so they don't get forgotten automatically.

To avoid manually creating move-constructors, which is error-prone, the \code{ForgetOnMove<T>} class wraps these values and automatically forgets them when the move-constructor is invoked.
The final result is shown in \cref{fig:ForgetOnMoveEx}: a clean, easy, and safe method of implementing new smart resource classes.
% Detecting and mitigating these faults and errors is an important part of effective programming, 
\begin{figurepage}
\begin{figure}
    \centering
    \begin{cppcode}
// Doesn't use ForgetOnMove<>
class ComplexWrapper {
    void* memoryPointer;
    
    // Constructor
    ComplexWrapper(size_t memorySize) {
        // Allocate memory
    }
    // Destructor
    ~ComplexWrapper() {
        if (memoryPointer != nullptr) {
            // Free memory
        }
    }
    
    // Copy constructor - deleted
    ComplexWrapper(const ComplexWrapper&) = delete;
    // Move constructor - complicated
    ComplexWrapper(ComplexWrapper&& movedFrom) {
        this->memoryPointer = movedFrom.memoryPointer;
        movedFrom.memoryPointer = nullptr;
    }
};

// Uses ForgetOnMove<>, is simpler
class SimpleWrapper {
    ForgetOnMove<void*> memoryPointer;
    
    // Constructor as before
    SimpleWrapper(size_t memorySize) {
        // Allocate memory
    }
    // Destructor as before, but checks if the memoryPointer is present
    ~SimpleWrapper() {
        if (memoryPointer.has_value()) {
            // Free memory
        }
    }
    
    // Copy constructor - deleted
    SimpleWrapper(const SimpleWrapper&) = delete;
    // Move constructor - defaulted!
    // We don't have to write this in full
    SimpleWrapper(SimpleWrapper&&) = default;
};
    \end{cppcode}
    \caption{Example of \code{ForgetOnMove<T>} vs. manual handling.}
    \label{fig:ForgetOnMoveEx}
\end{figure}
\end{figurepage}