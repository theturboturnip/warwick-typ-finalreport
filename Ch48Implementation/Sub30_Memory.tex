\section{Memory Layer}
As mentioned in the Design section, all elements of the Memory system are parameterized on the memory type.
This was accomplished by creating an enum \code{MType} and a set of classes templated on it (\cref{fig:TypeclassMemory}).
These templates were then specialized for each type of memory, implementing the logic for each type separately.
Separate implementations were required due to the unique constraints and allocation methods of each memory type.

\begin{figure}
    \centering
\begin{minted}{cpp}
enum MType {
    CPU,
    Cuda,
    VulkanCuda
};

enum RedBlackStorage {
    RedBlackOnly, // Just store the red and black matrices
    WithJoined // Store the red, black, and combined matrices
};

typeclass DataArray {
    // Has a static value MemType telling you what memory type it takes
    static MType MemType;
    // Has a function for calculating the total bytes used for a matrix
    static size_t sizeBytesOf(Size<uint32_t> size);
}

typeclass FrameAllocator<MType> {
    // Function for allocating a 2D array
    Sim2DArray<T, MType> allocate2D(Size);
    
    // Function for allocating a red/black array
    SimRedBlackArray<T, MType, RedBlackStorage> allocateRedBlack(Size);
}

typeclass FrameSetAllocator<MType, TFrame> {
    std::vector<TFrame> frames;
}
\end{minted}
    \caption{Memory Layer Typeclasses}
    \label{fig:TypeclassMemory}
\end{figure}

\subsection{Array Handles}
The \code{Sim2DArray<T, MType>} and \code{SimRedBlackArray<T, MType, RedBlackStorage>} classes represent handles to 2D arrays of values of type \code{T}.
They implement a common \emph{DataArray} typeclass (\cref{fig:TypeclassMemory}).
They do not own the data they point to, so if a \code{Sim2DArray} is destroyed the referenced memory isn't freed.
Freeing memory is handled by the \code{FrameAllocator} instead.

\code{SimRedBlackArray} serves as an aggregate of \code{Sim2DArray}s, without defining any special behaviour.
It does not specialize on the memory type, but provides two storage variants \code{RedBlackOnly} and \code{WithJoined} (\cref{fig:TypeclassMemory}).

\code{Sim2DArray} implements a set of functions for accessing data from the CPU, CUDA, and Vulkan.%GPU, and as Vulkan handles.
These functions are only present if that access type is supported, so trying to use an unsupported function results in a compile-time error.
Other more generic operations are also supported such as zeroing out memory, copying memory in from different sources, and copying memory out to the CPU.
Each of these is implemented differently based on the memory type, and may not be implemented if the operation is impossible.
For example attempting to `prefetch', i.e. move CUDA Unified Memory to the GPU, is only supported for CUDA Unified Memory and not the other types.
% \todomark{Matrix of supported operations}

\subsection{\texttt{FrameAllocator}}
\code{FrameAllocator<MType>} is an allocator associated with a single memory frame.
The behaviour is specialized for each memory type, but each specialization fits a typeclass (\cref{fig:TypeclassMemory}) for allocating 2D and red-black arrays.
The \code{FrameAllocator} owns these allocations, and when it is destroyed the allocations are freed.
The CPU and CUDA Managed memory variants allocate data directly using their respective allocation functions when requested, store the raw pointers in a list, and frees all of them on destruction.

The Vulkan variant allocates a fixed amount of Vulkan device memory, which is exported using the CUDA-Vulkan interop API to a CUDA-compatible pointer.
All new allocations are then sub-allocated from this memory.
The fixed amount is calculated by the \code{FrameSetAllocator}, which assumes only the bare minimum ($u, v, p, fluidmask$) requirements are allocated in Vulkan.
The associated CUDA pointer is \emph{not} compatible with Unified Memory, so cannot be paged to the CPU.

\subsection{\texttt{FrameSetAllocator}}
\code{FrameSetAllocator<MType, TFrame>} creates a set of TFrame objects which are each allocated using separate \code{FrameAllocator<MType>} objects.
The CPU and CUDA variants simply construct N \code{TFrame}s using N \code{FrameAllocator}s.
The Vulkan variant is more advanced as it has to check the \code{TFrame} exposes the correct data, calculate the amount of Vulkan data to allocate per frame, and expose this data by implementing a virtual \code{VulkanFrameSetAllocator} interface.
The Vulkan variant also passes in a CUDA \code{FrameAllocator} with the Vulkan one to allow the other buffers to be allocated.

\subsection{Usage in Other Layers}
The Simulation layer instantiates \code{FrameSetAllocator}s in the simulation Runners.
The Backend typeclass (\cref{fig:TypeclassBackend}) requires each backend to implement a Frame class, and to take a list of Frame instances as an argument to their constructor.

The visualization uses the \code{VulkanFrameSetAllocator} interface to grab references to simulation memory, which it uses while rendering. %and use these references while rendering.
