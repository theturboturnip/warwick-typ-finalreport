\section{Preliminary Work \& Background}
The primary languages used in the program are C++17 and CUDA.
This section will explain key elements of C++17 used in the program, the build system, and the external libraries used.

\subsection{C++ Primer}
Virtual classes use virtual functions to allow subclasses to override behaviour in the parent.
The seminal example is creating a parent class \code{Animal} which can \code{talk()}, and a subclass \code{Dog} which overrides \code{talk()} to bark.
When a virtual function is called on an object, instead of statically determining which function to call at compile-time, the \emph{vtable} of the object is read out at run-time with the correct function pointer\cite{presentation:RuntimePolymorphism}.
%Citation formerly \todocite{https://pabloariasal.github.io/2017/06/10/understanding-virtual-tables/}
In Java and Python all functions are considered virtual, but in C++ virtual behaviour can be selectively enabled.
As each virtual function call requires multiple indirections (object $\rightarrow$ vtable $\rightarrow$ function), the performance is slightly worse than using normal functions (see \cref{fig:virtual_inheritance}).
Virtual functions are avoided where possible in the codebase.

\begin{figure}[ht]
    \centering
    \begin{subfigure}{0.49\linewidth}%
        \begin{cppcode}
class Animal {
public:
    virtual void talk() = 0;

    virtual ~Animal() = default;
};

class Dog : public Animal {
public:
    void talk() override {
        printf("bark\n");
    }

    ~Dog() override = default;
};

int main() {
    // Virtual call
    Animal* animal = new Dog();
    animal->talk(); // Prints 'bark'
    delete animal;

    Dog dog = {};
    dog.talk();

    return 0;
}
        \end{cppcode}
        \caption{C++ implementation}
    \end{subfigure}%
    \begin{subfigure}{0.49\linewidth}%
        \begin{minted}{gas}
# Calling the function virtually (animal->talk())
mov     rax, QWORD PTR [rbp-24]
mov     rax, QWORD PTR [rax]
mov     rdx, QWORD PTR [rax]
mov     rax, QWORD PTR [rbp-24]
mov     rdi, rax
call    rdx
        
# Calling the function statically (dog.talk())
lea     rax, [rbp-32]
mov     rdi, rax
call    Dog::talk()
        \end{minted}
        \caption{x86 Assembly for calling the functions}
    \end{subfigure}%
    \caption{Inefficiencies of virtual inheritance\\{x86 assembly code generated from \url{https://godbolt.org/z/5hvE896aj}}}
    \label{fig:virtual_inheritance}
\end{figure}

One of C++'s greatest innovations over C is the template system.
Classes and functions can be `templated' on types or values, and then `instantiated' when these parameters are known.
When such a class or function is instantiated a complete copy is created with the new parameter values, which is compiled and optimized separately from any other instantiations.
This is useful for encoding extra information in a type for safety, e.g. \code{VulkanShader<Vertex>} cannot be passed to a function expecting \code{VulkanShader<Compute>} because they're independent types.
It's also useful for static function dispatch, as instead of taking a virtual class with a \code{talk()} function you can instead template a function on the type of animal it uses, and call the function directly.
This technique is used in the Simulation to efficiently use Backends.

\begin{figure}
    \centering
    \begin{subfigure}{0.49\linewidth}%
        \begin{cppcode}
class Dog {
public:
    void talk() {
        printf("bark\n");
    }
};

class Cat {
public:
    void talk() {
        printf("meow\n");
    }
};

template<class TAnimal>
void make_animal_talk(TAnimal* animal) {
    animal->talk();
}

int main() {
    Dog dog{};
    // Instantiates make_animal_talk<Dog>, which calls Dog::talk statically
    make_animal_talk<Dog>(&dog);

    Cat cat{};
    // Instantiates make_animal_talk<Cat>, which calls Cat::talk statically
    make_animal_talk<Cat>(&cat);

    return 0;
}
        \end{cppcode}
        % \caption{C++ implementation}
    \end{subfigure}%
    \begin{subfigure}{0.49\linewidth}%
        \begin{minted}{gas}
# void make_animal_talk<Dog>(Dog*):
# ...
mov     rax, QWORD PTR [rbp-8]
mov     rdi, rax
call    Dog::talk()
# ...
        
# void make_animal_talk<Cat>(Cat*):
# ...
mov     rax, QWORD PTR [rbp-8]
mov     rdi, rax
call    Cat::talk()
# ...
        \end{minted}
        % \caption{x86 Assembly for templated functions}
    \end{subfigure}%
    \caption{Using templates for polymorphism}
    \label{fig:templated_polymorphism}
    \todocite{\url{https://godbolt.org/z/hfM465EYa}}
\end{figure}


\subsubsection{``Typeclasses''}
In other languages, like Haskell, a typeclass defines some behaviour a class should fit. From \cite{learnyouahaskell}: ``If a type is a part of a typeclass, that means that it supports and implements the behavior the typeclass describes''.
C++17 does not have a convenient way of denoting this but it is especially helpful when building generic code with templates, as it allows the generic code to make assumptions about what behaviour types will support.
The rest of this chapter will define typeclasses where convenient to describe behaviour shared by certain classes.

\subsection{Build System}
The build system is implemented in CMake as specified in \cref{sec:ProjManagementTools}. %\cite{tool:Cmake}
This section highlights a few changes that were made to an otherwise standard setup to accommodate the project.

\subsubsection{CUDA-less Binaries}
The project can be built to produce both CUDA and CUDA-less binaries, in case it needs to be run on CUDA-less computers.
The list of regular C++ source files and CUDA source files are maintained separately. A CUDA-less binary (\shell{sim\_nocuda}) will only build the C++ files while a CUDA binary (\shell{sim\_cuda}) will build both.
When building the \shell{sim\_cuda} target the preprocessor macro \code{CUDA\_ENABLED} is defined throughout all source files, including the C++ files.
This allows support for CUDA backends in C++ code (i.e. as selectable options on the command-line) to be conditionally enabled without maintaining two copies of the relevant source files.
In \cref{fig:ConditionalCUDA} (which has been amended for brevity), the switch statement only contains a case for CUDA if the directive is set, triggering a fatal error otherwise.
\begin{figure}[ht]
    \centering
\begin{cppcode}
switch(backendType) {
    case Null:
        return SimFixedTimeRunner<NullSimulation, Host2DAllocator>();
    case CpuSimple:
        return SimFixedTimeRunner<CpuSimpleSimBackend, Host2DAllocator>();
    case CpuOptimized:
        return SimFixedTimeRunner<CpuOptimizedSimBackend, Host2DAllocator>();
    case CpuAdapted:
        return SimFixedTimeRunner<CpuOptimizedAdaptedSimBackend, Host2DAllocator>();
#if CUDA_ENABLED
    case CUDA:
        return SimFixedTimeRunner<CudaBackendV1<true>, CudaUnified2DAllocator>();
#endif
    default:
        FATAL_ERROR("Enum val %d doesn't have an ISimFixedTimeRunner!\n", backendType);
}\end{cppcode}
\caption{Conditionally supporting CUDA based on a preprocessor directive}
    \label{fig:ConditionalCUDA}
\end{figure}


% \todomark{Move Shader Build Infrastructure somewhere else?}
\subsubsection{Shader Build Infrastructure} % Fits in with the build system vibe
The shaders used for visualization are written in GLSL, with appropriate extensions to be compatible with Vulkan.
They are separated by file type, with Vertex shaders in \shell{.vert} files and Fragment shaders in \shell{.frag} files.
As Vulkan does not natively support GLSL, they must be compiled to SPIR-V before they can be used.
CMake does not support GLSL as a first-class language, so a custom build command was used to compile them with \shell{glslc}\cite{GoogleLLCShaderc} when they change.
This allows them to be treated just like any other source file from the programmer's perspective.
The SPIR-V files are placed in a \shell{shaders} directory next to the binaries, where they can be easily accessed and passed to Vulkan.

% \todomark{Code Safety - emphasis on using templates, compile-time checks, and failing that runtime assertions.}

\subsection{Library Selection}
\label{sec:LibrarySelection}
\begin{figure}[ht]
    \centering
    \begin{tabular}{r|c c}
        & OpenGL & Vulkan \\
        \hline
        OpenCL & Y & N \\ 
        CUDA & Y & Y \\ 
        OpenGL & Y & N \\ 
        Vulkan & N & Y \\ 
    \end{tabular}
    \caption{Graphics and Compute Backend Interoperability Matrix}
    \label{fig:LibraryChoices}
\end{figure}
CUDA and Vulkan have been chosen as backends, but other backends were also considered.
As the simulation would have to run on DCS systems (\cref{reqN:DCSCompile}) and thus run on Linux, the only possible GPU rendering backends were OpenGL and Vulkan.
However, there were still multiple choices of compute backend:
\begin{itemize}
    \item OpenCL\cite{tool:OpenCL1.0PressRelease} is an ``Open Standard for Parallel Programming of Heterogeneous Systems''\cite{TheKhronosGroupOpenCLInc}.
    \item CUDA\cite{tool:CUDA} is a closed-source library for running parallel code on NVIDIA GPUs.
    \item OpenGL has Compute Shaders\cite{tool:OpenGLComputeShaderExt} which can execute computations outside of the graphics pipeline.
    \item Vulkan also has Compute capability\cite{TheKhronosGroupVulkanGuide}, similar in function to OpenGL.
\end{itemize}
To decide on the compute backend to use, an interoperability matrix was drawn (\cref{fig:LibraryChoices}) to show which libraries could share data without copying it between buffers.
As the researcher was already experienced with Vulkan, and the more granular control it provides would be beneficial to performance, Vulkan was selected as the rendering backend.
This prevented OpenCL and OpenGL from being used as compute backends, as they are not compatible with Vulkan.
CUDA and Vulkan have comparable ability, but CUDA was chosen as the compute backend.
The Vulkan compute shaders are still a very graphics-oriented view of computation, and CUDA would give the researcher experience with other kinds of libraries.
A Vulkan compute backend is used for the visualization portion of the code.

% Not for now, but note that the C++ vulkan bindings are nice. They do end up wrapped in other classes, but they remove the need to remember boilerplate as much.

In other cases, there were clear choices: the SDL2\cite{SimpleHomepage} window and input library and the Dear ImGUI\cite{CornutDearImGui} UI library were chosen due to personal experience.
The \shell{stb\_image.h} header was found to be a simple method of importing image colour data as byte arrays, used for the input generator (\cref{req:GenerateState}).

There are a great many options for Command-Line parsing libraries, even more so because C++ is used instead of C.
A recent survey of the possibilities\cite{attractivechaos2018AC/C++} was whittled down to five options.

\code{getopt}\cite{FreeSoftwareFoundationGetopt3:Page}, \code{argp}\cite{GNUProjectArgpLibrary}, and \code{gopt}\cite{VajzovicGoptLibrary} are C libraries that use arrays of structures to define the required arguments.
Of them, only \code{argp} can automatically generate a \shell{--help} argument, which is a very valuable feature.
\code{cxxopts}\cite{jarro2783Cxxopts:Parser} was considered as a C++ alternative but used very odd syntax for defining arguments.
Ultimately CLI11\cite{CLIUtilsCLI11} was chosen as a modern C++11 library that had native support for subcommands, which were used heavily for separating program components (see \cref{sec:DesignSubcommands}).

% \todomark{Resource Management}