\begin{figure}[ht]
    \centering
    \begin{subfigure}{0.49\linewidth}%
        \begin{cppcode}
class Animal {
public:
    virtual void talk() = 0;

    virtual ~Animal() = default;
};

class Dog : public Animal {
public:
    void talk() override {
        printf("bark\n");
    }

    ~Dog() override = default;
};

int main() {
    // Virtual call
    Animal* animal = new Dog();
    animal->talk(); // Prints 'bark'
    delete animal;

    Dog dog = {};
    dog.talk();

    return 0;
}
        \end{cppcode}
        \caption{C++ implementation}
    \end{subfigure}%
    \begin{subfigure}{0.49\linewidth}%
        \begin{minted}{gas}
# Calling the function virtually (animal->talk())
mov     rax, QWORD PTR [rbp-24]
mov     rax, QWORD PTR [rax]
mov     rdx, QWORD PTR [rax]
mov     rax, QWORD PTR [rbp-24]
mov     rdi, rax
call    rdx
        
# Calling the function statically (dog.talk())
lea     rax, [rbp-32]
mov     rdi, rax
call    Dog::talk()
        \end{minted}
        \caption{x86 Assembly for calling the functions}
    \end{subfigure}%
    \caption{Inefficiencies of virtual inheritance}
    \label{fig:virtual_inheritance}
    \todocite{\url{https://godbolt.org/z/5hvE896aj}}
\end{figure}