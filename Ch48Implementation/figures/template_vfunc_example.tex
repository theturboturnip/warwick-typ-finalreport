\begin{figure}
    \centering
    \begin{subfigure}{0.49\linewidth}%
        \begin{cppcode}
class Dog {
public:
    void talk() {
        printf("bark\n");
    }
};

class Cat {
public:
    void talk() {
        printf("meow\n");
    }
};

template<class TAnimal>
void make_animal_talk(TAnimal* animal) {
    animal->talk();
}

int main() {
    Dog dog{};
    // Instantiates make_animal_talk<Dog>, which calls Dog::talk statically
    make_animal_talk<Dog>(&dog);

    Cat cat{};
    // Instantiates make_animal_talk<Cat>, which calls Cat::talk statically
    make_animal_talk<Cat>(&cat);

    return 0;
}
        \end{cppcode}
        % \caption{C++ implementation}
    \end{subfigure}%
    \begin{subfigure}{0.49\linewidth}%
        \begin{minted}{gas}
# void make_animal_talk<Dog>(Dog*):
# ...
mov     rax, QWORD PTR [rbp-8]
mov     rdi, rax
call    Dog::talk()
# ...
        
# void make_animal_talk<Cat>(Cat*):
# ...
mov     rax, QWORD PTR [rbp-8]
mov     rdi, rax
call    Cat::talk()
# ...
        \end{minted}
        % \caption{x86 Assembly for templated functions}
    \end{subfigure}%
    \caption{Using templates for polymorphism\\(x86 assembly code generated from \url{https://godbolt.org/z/hfM465EYa})}%
    \label{fig:templated_polymorphism}%
\end{figure}
