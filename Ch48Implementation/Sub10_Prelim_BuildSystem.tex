\subsection{Build System}
The build system is implemented in CMake as specified in \cref{sec:ProjManagementTools}. %\cite{tool:Cmake}
This section highlights a few changes that were made to an otherwise standard setup to accommodate the project.

\subsubsection{CUDA-less Binaries}
The project can be built to produce both CUDA and CUDA-less binaries, in case it needs to be run on computers without CUDA.
The list of regular C++ source files and CUDA source files are maintained separately. A CUDA-less binary (\shell{sim\_nocuda}) will only build the C++ files while a CUDA binary (\shell{sim\_cuda}) will build both.
When building with CUDA support the preprocessor macro \code{CUDA\_ENABLED} is defined in all source files, including the C++ files.
This allows support for CUDA backends in C++ code (i.e. as selectable options on the command-line) to be conditionally enabled without maintaining two copies of the relevant source files.
In \cref{fig:ConditionalCUDA} (which has been amended for brevity), the switch statement only contains a case for CUDA if the directive is set, triggering a fatal error otherwise.
The enumeration defining the existing Backends also uses this technique to completely remove the concept of a CUDA backend from non-CUDA builds.
\begin{figure}[ht]
    \centering
\begin{cppcode}
switch(backendType) {
    case Null:
        return SimFixedTimeRunner<NullSimulation, Host2DAllocator>();
    case CpuSimple:
        return SimFixedTimeRunner<CpuSimpleSimBackend, Host2DAllocator>();
    case CpuOptimized:
        return SimFixedTimeRunner<CpuOptimizedSimBackend, Host2DAllocator>();
    case CpuAdapted:
        return SimFixedTimeRunner<CpuOptimizedAdaptedSimBackend, Host2DAllocator>();
#if CUDA_ENABLED
    case CUDA:
        return SimFixedTimeRunner<CudaBackendV1<true>, CudaUnified2DAllocator>();
#endif
    default:
        FATAL_ERROR("Enum val %d doesn't have an ISimFixedTimeRunner!\n", backendType);
}\end{cppcode}
\caption{Conditionally supporting CUDA based on a preprocessor directive}
    \label{fig:ConditionalCUDA}
\end{figure}


% \todomark{Move Shader Build Infrastructure somewhere else?}
\subsubsection{Shader Build Infrastructure} % Fits in with the build system vibe
The shaders used for visualization are written in GLSL, with appropriate extensions to be compatible with Vulkan.
They are separated by file type, with Vertex shaders in \shell{.vert} files, Fragment shaders in \shell{.frag} files, and Compute shaders in \shell{.comp} files.
As Vulkan does not natively support GLSL, they must be compiled to SPIR-V before they can be used.
CMake does not support GLSL as a first-class language, so a custom build command was used to compile them with \shell{glslc}\cite{GoogleLLCShaderc} when they change.
This allows them to be treated just like any other source file from the programmer's perspective.
SPIR-V files are placed in a \shell{shaders} directory next to the binaries, where they can be easily accessed and passed to Vulkan.