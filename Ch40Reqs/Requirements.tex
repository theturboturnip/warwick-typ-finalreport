% !TEX root =  ../FinalReport.tex

\chapter{Project Requirements}
\label{sec:Requirements}
These basic functional and non-functional requirements define the baseline the final result will be measured against.
\todomark{Elaborate on requirements}

\newcommand{\must}[0]{\textbf{must}}
\newcommand{\should}[0]{\textbf{should}}
\newcommand{\shouldnt}[0]{\textbf{should not}}

\section{Functional Requirements}
\begin{reqF}
    \item \label{req:StoreState} The system \must{} store simulation state in a file or set of files.
    \item \label{req:LoadState} The system \must{} be able to load the initial state of a simulation from these file(s).
    \item \label{req:GenerateState} The system \must{} be able to generate initial simulation state files.
    \item \label{req:HeadlessSim} The system \must{} be able to simulate from an initial state for a set amount of time without visualizing.
    \begin{reqF}
        \item \label[reqFi]{req:HeadlessOutput} This mode \must{} be able to store the final state to output file(s).
    \end{reqF} 
    \item \label{req:VizSim} The system \must{} be able to simulate from an initial state for an indeterminate amount of time while visualizing.
    \begin{reqF}
        \item \label[reqFi]{req:VizPauseResume} This mode \must{} allow the user to pause and resume the simulation.
        \item \label[reqFi]{req:VizSaveState} This mode \should{} be able to save it's state to output file(s) when requested.
        \item \label[reqFi]{req:VizManip} This mode \should{} allow the user to manipulate the simulation or visualization state while simulating.
        \item \label[reqFi]{req:VizLockedFPS} This mode \should{} be able to run at a locked frame-rate.
        \item \label[reqFi]{req:VizFlatOut} This mode \should{} be able to run as fast as possible, without locking the framerate.
        \item \label[reqFi]{req:VizSomeSpeed} This mode \must{} be able to perform at least one of \cref{req:VizLockedFPS,req:VizFlatOut}.
    \end{reqF} 
    \item \label{req:GPUCapable} Both methods of simulation \must{} be capable of using the GPU for simulating.
    \item \label{req:Compare} The system \must{} be able to compare how similar two simulation states are.
    \begin{reqF}
        \item \label[reqFi]{req:CompareBinary} This comparison \should{} produce a binary SIMILAR/NOT~SIMILAR verdict using heuristics.
    \end{reqF}
    
    \item \label{req:VizLayers} The visualization \must{} consist of multiple layers which can be individually controlled.
    \begin{reqF}
        \item \label[reqFi]{req:VizLayersBackground} The visualization \must{} always display a background layer which shows the simulation obstacles in a different color to the fluid.
        \item \label[reqFi]{req:VizLayersScalar} The visualization \must{} be able to display an optional scalar quantity (e.g. pressure) using a color scale, where the value is within a user-defined range.
        \item \label[reqFi]{req:VizLayersVector} The visualization \must{} be able to display an optional vector quantity (e.g. velocity) using a vector field, where the magnitude is within a user-defined range.
        \item \label[reqFi]{req:VizLayersParticle} The visualization \must{} feature an optional particle simulation, where particles are continuously emitted and move with the velocity of the field.
    \end{reqF}
    \item \label{req:VizAutoRange} The scalar and vector quantities (\cref{req:VizLayersScalar,req:VizLayersVector}) \should{} have an auto-range function to automatically calculate the range based on the values present.
    \item \label{req:VizColors} All colors used in the visualization (e.g. particle colors, the scalar color scale) \should{} be user-controlled.
    \item \label{req:VizParticleEmission} The locations of particle emitters \should{} be user-controllable.
\end{reqF}

\pagebreak
\section{Non-Functional Requirements}
\begin{reqNF}
    \item \label{reqN:LargeData} The system \must{} be capable of operating on large datasets (e.g. 4096x4096 grids) without failing.
    \item \label{reqN:Resources} The system \must{} be efficient and avoid wasting any resources allotted to it.
    \item \label{reqN:SimilarOutput} The simulation \must{} produce similar results to the original coursework when equivalent initial state is used.
    \item \label{reqN:SimSpeed} The simulation \should{} run at least 2x as fast as the original coursework when equivalent initial state is used.
    \item \label{reqN:Realtime} The visualized simulation \must{} run in real-time at framerates~$\ge$~30 FPS for some outputs.
    \item \label{reqN:VizSpeed} The visualization features \shouldnt{} have a significant impact on the framerate.
    \item \label{reqN:Intuitive} The visualized simulation \should{} intuitively represent the fluid flow such that it can be understood by someone unfamiliar with fluid simulation.
    \item \label{reqN:VizParticleAdvanced} The particle simulation (\cref{req:VizLayersParticle}) \should{} demonstrate advanced behaviour to make the visualization more intuitive e.g. avoiding clumping.
    \item \label{reqN:Documented} The system \must{} be fully documented and maintainable. % TODO Wheeler referenced BCS code of conduct again??
    \item \label{reqN:UsageGuide} The system \should{} have a simple guide to common operations for new users to refer to.
    \item \label{reqN:DCSCompile} The system \should{} be fully compilable and executable from a DCS machine with minimal extra installations.
\end{reqNF}

\section{Hardware and Software Constraints}
\label{sec:Requirements_HardwareSoftware}
As this simulation uses a GPU, the developer must have one available for debugging and testing the program.
As the CUDA API is used to implement the simulation (see \cref{sec:LibrarySelection}), the program requires an NVIDIA GPU to run.

The high-speed rendering requirements of the program necessitated the use of Vulkan over OpenGL.
Vulkan gives the developer more fine control over scheduling, and allows the hardware to take shortcuts that it may not be able to do under OpenGL.
For more on this decision see \cref{sec:LibrarySelection}.