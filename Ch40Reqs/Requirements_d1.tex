% !TEX root =  ../FinalReport.tex

\chapter{Project Requirements}
\label{sec:Requirements}
These basic functional and non-functional requirements define the baseline the final result will be measured against.
These are mostly unchanged from the Specification document, and may be expanded upon in the final report as the project evolves and more testable features are added.
\cref{req:VizLockedFPS,req:VizFlatOut,req:VizSomeSpeed} have been added as the Specification was unclear as to the speed of the visualization.
Functional requirements for visualization are intentionally not included, as an intuitive visualization can take many forms that must be investigated further before being specified.

\newcommand{\must}[0]{\textbf{must}}
\newcommand{\should}[0]{\textbf{should}}

\section{Functional Requirements}
\todopending{Remove the need for [reqFi] here, it stops Overleaf from detecting requirements}
\begin{reqF}
    \item \label{req:StoreState} The system \must{} store simulation state in a file or set of files.
    \item \label{req:LoadState} The system \must{} be able to load the initial state of a simulation from these file(s).
    \item \label{req:GenerateState} The system \must{} be able to generate initial simulation state files.
    \item \label{req:HeadlessSim} The system \must{} be able to simulate from an initial state for a set amount of time without visualizing.
    \begin{reqF}
        \item \label[reqFi]{req:HeadlessOutput} This mode \must{} be able to store the final state to output file(s).
    \end{reqF} 
    \item \label{req:VizSim} The system \must{} be able to simulate from an initial state for an indeterminate amount of time while visualizing.
    \begin{reqF}
        \item \label[reqFi]{req:VizPauseResume} This mode \must{} allow the user to pause and resume the simulation.
        \item \label[reqFi]{req:VizSaveState} This mode \should{} be able to save it's state to output file(s) when requested.
        \item \label[reqFi]{req:VizManip} This mode \should{} allow the user to manipulate the simulation state while simulating.
        \item \label[reqFi]{req:VizLockedFPS} This mode \should{} be able to run at a locked frame-rate.
        \item \label[reqFi]{req:VizFlatOut} This mode \should{} be able to run as fast as possible, without locking the framerate.
        \item \label[reqFi]{req:VizSomeSpeed} This mode \must{} be able to perform at least one of \cref{req:VizLockedFPS,req:VizFlatOut}.
    \end{reqF} 
    \item \label{req:GPUCapable} Both methods of simulation \must{} be capable of using the GPU for simulating.
    \item \label{req:Compare} The system \must{} be able to compare how similar two simulation states are.
    \begin{reqF}
        \item \label[reqFi]{req:CompareBinary} This comparison \should{} produce a binary SIMILAR/NOT~SIMILAR verdict using heuristics.
    \end{reqF}
\end{reqF}

\section{Non-Functional Requirements}
\begin{reqNF}
    \item \label{reqN:SimilarOutput} The simulation \must{} produce similar results to the original coursework when equivalent initial state is used.
    \item \label{reqN:Realtime} The visualized simulation \must{} run in real-time at framerates~$\ge$~30 FPS for some outputs.
    \item \label{reqN:Intuitive} The visualized simulation \should{} intuitively represent the fluid flow such that it can be understood by someone unfamiliar with fluid simulation.
    \item \label{reqN:Documented} The system \must{} be fully documented and maintainable. % TODO Wheeler referenced BCS code of conduct again??
    \item \label{reqN:UsageGuide} The system \should{} have a simple guide to common operations for new users to refer to.
    \item \label{reqN:LargeData} The system \must{} be capable of operating on large datasets (e.g. 4096x4096 grids) without failing.
    \item \label{reqN:DCSCompile} The system \should{} be fully compilable and executable from a DCS machine with minimal extra installations.
\end{reqNF}

\todopending{Coverage matrix?}

\section{Hardware and Software Constraints}
\label{sec:Requirements_HardwareSoftware}
As this simulation uses a GPU, the developer must have one available for debugging and testing the program.
As the CUDA API is used to implement the simulation (see \cref{sec:LibrarySelection}), the program requires an NVIDIA GPU to run.

The high-speed rendering requirements of the program necessitated the use of Vulkan over OpenGL.
Vulkan gives the developer more fine control over scheduling, and allows the hardware to take shortcuts that it may not be able to do under OpenGL.
For more on this decision see \cref{sec:LibrarySelection}.