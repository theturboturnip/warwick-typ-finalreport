% !TEX root =  ../FinalReport.tex

\chapter{Testing}
\label{sec:Testing} 
In order to measure the degree of success a project achieves, testing must be performed to verify the behaviour of the program is correct.
Building tests also allows further development of the program to easily identify when new bugs are introduced.
% The tests proposed in this section have been implemented for these purposes.
This section proposes potential effective tests, and documents the results of tests already performed since they were described in the Specification.


\section{Unit Testing}
The separate phases of the simulation are effective units of code.
They could be automatically tested individually, or individual units could be swapped out for known working versions in order to pinpoint bugs found in wider tests.
The latter method has been used for debugging during the simulation implementation.

The \texttt{makeinput} (\cref{req:GenerateState}), \texttt{compare} (\cref{req:Compare}), and \texttt{renderppm} program modes can also be tested as individual units with input/expected output combinations.
These tests have not yet been implemented, but are planned as an extension.\label{sec:TestsForSubcommands}

\section{Integration Testing}
The ``headless mode'' outlined in \cref{req:HeadlessSim} has functioned as an integration test for all of the simulation phases.
Initially the CPU Simple and Optimized backends (\cref{sec:DesignBackends}) were added and tested against the original ACA program\cite{modules:CS257Coursework} and the submitted coursework\cite{modules:aca257submission} using the provided testing tools.
The Compare mode (\cref{req:Compare}) was then implemented and tested against the ACA testing tools to ensure it's behaviour was correct.
The Optimized Adapted and CUDA backends (\cref{sec:DesignBackends}) were then added and compared to the required output to ensure that any deviations were small.\footnote{Because the simulation operates on single-precision floating point numbers, small changes to orders of operation or compiler optimizations could introduce small discrepancies at the bit level.}

While the visualization cannot be tested without some simulation data to visualize, that data does not necessarily need to be continuously simulated.
Static simulation states may be created in order to test separate parts of the visualization, or multiple parts at once.\label{sec:TestsForViz}

\section{Overall Testing}
The ``visualization mode'' from \cref{req:VizSim} should function as a full system test of the simulation with the visualization.
Assuming the headless simulations are accurate, there should be a negligible difference in results from a visualized simulation.
