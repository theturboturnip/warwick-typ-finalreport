% !TEX root =  ../FinalReport.tex

\chapter{Evaluation}\label{sec:Evaluation}

This chapter evaluates the project's end result against the requirements laid out in \cref{sec:Requirements}, providing an objective measurement of success.
This is then supplemented with a justification for the few missed requirements, all of which were non-essential.
The project management is then appraised.

\section{Requirements Evaluation}
The Requirements in \cref{sec:Requirements} evolved as the simulation and visualization areas were researched throughout the project.
The program was designed with these requirements in mind, and by checking if the program meets these requirements the overall success of the project can be determined.
Requirements are prioritized as either \must{}-have or \should{}-have.
All \must{}-have requirements have been met, but three \should{}-have requirements have not (see \cref{tab:failed_req}).
These are justified in \cref{sec:Evaluation:FailedReq}.

Functional Requirements define the actions a program must/should be able to perform, such as \cref{req:GenerateState} stating ``the system must be able to generate initial simulation states''.
The functional requirements, the tests performed to check them, and the outcomes have been listed in \cref{tab:functional_req}.

Non-functional Requirements do not define actions, but rather define properties of those actions that must be met.
As an example \cref{reqN:SimSpeed} states that with the same initial input, the CUDA simulation ``should run at least 2x as fast as the original coursework''.
Similarly to functional requirements, these are laid out in \cref{tab:nonfunctional_req}.

\subsection{Failed Requirements}\label{sec:Evaluation:FailedReq}
\begin{table}[h]
    \centering
    \begin{tabular}{l|c|l|c}%ccl|p{0.4\linewidth}|m{0.2\linewidth}|c}
        ID & Priority & Tests & Status \\
        \hline
        \ref{req:VizSaveState} & \should{} & \ref{test:sys:run:save} & \testfail{}     \\
        \ref{req:CompareBinary} & \should{} & \ref{test:unit:compare:identical}, \ref{test:unit:compare:different} & \testfail{}      \\
        \hline
        \ref{reqN:VizParticleAdvanced} & \should{} & \ref{test:sys:run:layerPerms} & \testfail{}           \\
    \end{tabular}
    \caption{Failed Requirements}
    \label{tab:failed_req}
\end{table}

Of the thirty-five \must{}-have requirements, 100\% of them have been met.
Just three \should{}-have requirements failed, and each omission can be justified.
% The three failed requirements (\cref{tab:failed_req}) are all \should{}-haves, and each omission can be justified.
% These requirements are all \should{}-haves, and each omission can be justified.

Satisfying (\cref{req:VizSaveState}), which required saving the simulation state during a visualization, would have delayed the work on the visualization layers.
As the visualization is a significant portion of the project, and this feature would not have had a relevant use-case during development, it was cut.

\cref{req:CompareBinary} would require the \texttt{compare} tool to output a single SIMILAR/NOT~SIMILAR value rather than the more detailed metrics it now uses.
Using a single numeric value to convey this information would be near-impossible and would result in a significant loss of nuance, even more so if only binary options are available, and would be pointless if shown alongside the more detailed metrics.

\cref{reqN:VizParticleAdvanced} was planned in the Progress Report, specifically citing BOID-like behaviour\cite{BOIDS_10.1145/37401.37406} as a potential means for reducing clumping.
This would have greatly increased the complexity of the particle system.
Particles would need to identify other nearby particles, which would likely require the positions to be sorted for efficient access, and while this has been implemented on the GPU before\cite{Lindqvist_2018}\todocite{https://github.com/Shinao/Unity-GPU-Boids} it wasn't possible to implement it before the code freeze.
% All of these requirements are \should{}-haves, meaning the core elements of the project have been successful.
% Nevertheless these failures are 

\section{Project Management}
The Project Management has been incredibly successful, allowing an extremely complex combined simulation and visualization to be efficiently completed on-schedule.
The schedule itself allotted enough time for both research and implementation, and gave enough leeway that the manifested risks did not prevent success.
Using third-party libraries for GUI management and command-line parsing allowed developer time to be spent on important problems, instead of hooking together customized implementations.
The Code Freeze implemented in Week 22 ensured enough time was devoted to developing the presentation, which was crucial to it's success.

% \todomark{No Researcher's Evaluation here, tie those questions into the conclusion}
% \section{Researcher's Evaluation}
% % Complementing the objective analysis, this section includes a personal

% \\\\
% \textbf{What is the contribution of this project?}


% \\\\
% \textbf{Why was this a challenging project suitable to a Computer Systems Engineering degree?}

% \\\\
% \textbf{}

\section{Researcher's Evaluation}
Given the great success of the 