% !TEX root =  ../FinalReport.tex

\chapter{Evaluation}

\section{Requirements Evaluation}




\begin{table}[p]
    \centering
    \begin{tabular}{l|c|l|c}%ccl|p{0.4\linewidth}|m{0.2\linewidth}|c}
        ID & Priority & Tests & Status \\
        \hline
        \ref{req:StoreState} & \must{} & Implicit in program behaviour & \testsuccess{} \\
        \ref{req:LoadState} & \must{} & \ref{test:intg:input:sim}, \ref{test:intg:input:viz}, \ref{test:intg:sim:sim}, \ref{test:intg:sim:viz} & \testsuccess{}\\
        \ref{req:GenerateState} & \must{} & \ref{test:unit:makeinput} & \testsuccess{} \\
        \hline
        \ref{req:HeadlessSim} & \must{} & \ref{test:unit:fixedtime} & \testsuccess{}     \\
        \ref{req:HeadlessOutput} & \must{} & \ref{test:unit:fixedtime} & \testsuccess{}  \\
        \hline
        \ref{req:VizSim} & \must{} & \ref{test:unit:run} & \testsuccess{}     \\
        \ref{req:VizPauseResume} & \must{} & \ref{test:sys:run:pause} & \testsuccess{}     \\
        \ref{req:VizSaveState} & \should{} & \ref{test:sys:run:save} & \testfail{}     \\
        \ref{req:VizManip} & \should{} & \ref{test:sys:run:manip} & \testsuccess{}     \\
        \ref{req:VizLockedFPS} & \should{} & \ref{test:sys:run:lockedFPS} & \testsuccess{}     \\
        \ref{req:VizFlatOut} & \should{} & \ref{test:sys:run:flatoutFPS} & \testsuccess{}     \\
        \ref{req:VizSomeSpeed} & \must{} & \ref{test:sys:run:lockedFPS}, \ref{test:sys:run:flatoutFPS} & \testsuccess{}     \\
        \hline
        \ref{req:GPUCapable} & \must{} & \ref{test:sys:sim:gpu}, \ref{test:sys:run:gpu} & \testsuccess{}     \\
        \hline
        \ref{req:Compare} & \must{} & \ref{test:unit:compare:identical}, \ref{test:unit:compare:different} & \testsuccess{}     \\
        \ref{req:CompareBinary} & \should{} & \ref{test:unit:compare:different}, \ref{test:unit:compare:different} & \testfail{}      \\
        \hline
        \ref{req:VizLayers} & \must{} & \ref{test:sys:run:layerPerms} & \testsuccess{}     \\
        \ref{req:VizLayersBackground}  & \must{} & \ref{test:sys:run:layerPerms} & \testsuccess{}  \\
        \ref{req:VizLayersScalar}  & \must{} & \ref{test:sys:run:layerPerms} & \testsuccess{}      \\
        \ref{req:VizLayersVector}  & \must{} & \ref{test:sys:run:layerPerms} & \testsuccess{}      \\
        \ref{req:VizLayersParticle}  & \must{} & \ref{test:sys:run:layerPerms} & \testsuccess{}       \\
        \hline
        \ref{req:VizAutoRange} & \should{} & \ref{test:sys:run:autorange} & \testsuccess{}     \\
        \ref{req:VizColors} & \should{} & \ref{test:sys:run:colors}  & \testsuccess{}    \\
        \ref{req:VizParticleEmission} & \should{} & \ref{test:sys:run:manip} & \testsuccess{} \\
    \end{tabular}
    \caption{Evaluation of Functional Requirements}
    \label{tab:req_matrix_f}
\end{table}

% 24 total

\begin{table}[p]
    \centering
    \begin{tabular}{l|c|l|c}%ccl|p{0.4\linewidth}|m{0.2\linewidth}|c}
        ID & Priority & Tests & Status \\
        \hline
        \ref{reqN:LargeData} & \must{} & \ref{test:sys:sim:large} & \testsuccess{}          \\
        \ref{reqN:Resources} & \must{} & \ref{test:sys:run:validation}, \ref{test:sys:sim:valgrind}, \ref{test:sys:sim:cudamemcheck}, \ref{test:sys:run:pipeline} & \testsuccess{}          \\
        \ref{reqN:SimilarOutput} & \must{} & \ref{test:sys:sim:accuracy} & \testsuccess{}          \\
        \ref{reqN:SimSpeed} & \should{} & \ref{test:sys:sim:speed} & \testsuccess{}           \\
        \ref{reqN:Realtime} & \must{} & \ref{test:sys:run:highFPS} & \testsuccess{}           \\
        \ref{reqN:VizSpeed} & \should{} & \ref{test:sys:run:vizSpeed} & \testsuccess{}          \\
        \ref{reqN:Intuitive} & \should{} & By Inspection & \testsuccess{}         \\
        \ref{reqN:VizParticleAdvanced} & \should{} & \ref{test:sys:run:layerPerms} & \testfail{}           \\
        \ref{reqN:Documented} & \must{} & By Inspection & \todomark{\testsuccess{}}          \\
        \ref{reqN:UsageGuide} & \should{} & By Inspection & \todomark{\testsuccess{}}          \\
        \ref{reqN:DCSCompile} & \should{} & \todomark{DCS compile test} & \todomark{DCS compile result}          \\
    \end{tabular}
    \todomark{Very odd that \ref{test:sys:run:layerPerms} can be both positive for some layers and negative for others. Comparison SIMILAR/NOT SIMILAR tests are the same. Should add specific tests for each.}
    \caption{Evaluation of Non-Functional Requirements}
    \label{tab:nonfunctional_req}
\end{table}

% 11 total


\section{Optimizations}
The CUDA program attempts to arrange the threads such that they coalesce accesses, but it has not been verified to work yet.
\todomark{Verify}

Inspection of the compiled PTX\todocite{\url{https://docs.nvidia.com/cuda/parallel-thread-execution/index.html}} text shows that values are loaded using \code{ld.global} instructions which load from the global memory space\todocite{\url{https://docs.nvidia.com/cuda/parallel-thread-execution/index.html\#data-movement-and-conversion-instructions-ld}}.

% As this project is focused on improving upon the accuracy of the ACA submission, instead of producing bit-identical results, these optimizations have been implemented in different forms.
Fused multiply-add is inserted in many places by the CUDA compiler in Release mode\todocite{NVCC inserts FMA?}, single-precision floats are used in all cases, and the residual phase is still skipped.
\todomark{Comment on skipped residual phase, want to re-add but would need more stuff to be implemented for any benefit}
\footnote{The CUDA program still lacks a residual phase, but this is planned to be implemented later.\label{sec:OptimizationReAddingResidual}}

\todomark{Touch on FMA difference (\#pragma STDC FP_CONTRACT ON could help with the discrepancy?)}

\section{Project Management}

\todomark{not-tightly-coupled, because it allows easier parallelism}


