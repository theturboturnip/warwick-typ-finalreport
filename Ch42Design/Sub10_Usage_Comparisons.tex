\subsection{Comparison Heuristics}\label{sec:Comparisons}
In the comparison subcommand heuristics are used to judge if one simulation is accurate and precise with respect to the other.
This does not quite fulfil \cref{req:CompareBinary}, as there are two results and two heuristics used instead of just one, but it is useful for comparisons regardless so was not changed.

This assumes one of the supplied states is a known-valid simulation state, and the other is not.
The velocity and pressure values $u, v, p$ of two simulation states are compared separately.
The simulation states must be of the same resolution, and should use the same boundary squares (although this is not checked).

The comparison is performed by calculating the mean and standard deviation of the square error between the datasets.
These are then compared to tolerance values to produce two binary outputs: ACCURATE if the mean is below tolerance, and PRECISE if the standard deviation is below tolerance.
Examples are shown in \cref{fig:example_comparisons}.

The tolerance for the mean was derived from an expected error magnitude of $\pm 10^{-7}$, which was squared to produce $10^{-14}$.
It is assumed that the standard deviation should always be smaller than the mean, so the tolerance for standard deviation is also $10^{-14}$.
\todopending{More examination of this? Square error always >0 => if std.dev > mean then distribution cannot be normal? Prob wait for full report}

\begin{figure}[ht]
    \centering
    \begin{center}
    \lstset{language=bash,
    numbers=none,
    keywordstyle=[2]\color{red},
    keywords=[2]{INACCURATE,IMPRECISE},
    keywordstyle=[3]\color{green},
    keywords=[3]{ACCURATE,PRECISE}
    }
    % \lstset{language=bash,
    % linewidth=0.5\linewidth
    % % keywordstyle=[2]\color{red},
    % % keywords=[2]{INACCURATE,IMPRECISE},
    % % keywordstyle=[3]\color{green},
    % % keywords=[3]{ACCURATE,PRECISE}
    % }
    % \subcaptionbox{Comparison of equal states}[0.5\linewidth]{\begin{frame}[fragile]\end{frame}}
    %[xleftmargin=-0.2\textwidth]
\begin{subfigure}{0.4\textwidth}\begin{lstlisting}
Velocity X:
    Sq. Error Mean:     0     ACCURATE
    Sq. Error Std. Dev: 0     PRECISE
Velocity Y:
    Sq. Error Mean:     0     ACCURATE
    Sq. Error Std. Dev: 0     PRECISE
Pressure:
    Sq. Error Mean:     0     ACCURATE
    Sq. Error Std. Dev: 0     PRECISE\end{lstlisting}%
    \caption{Comparison of Equal States}%
\end{subfigure}%
\hspace{0.02\textwidth}
\begin{subfigure}{0.5\textwidth}\begin{lstlisting}[]
Velocity X:
    Sq. Error Mean:     0.0233842     INACCURATE
    Sq. Error Std. Dev: 0.0996487     IMPRECISE
Velocity Y:
    Sq. Error Mean:     0.00566354    INACCURATE
    Sq. Error Std. Dev: 0.0139529     IMPRECISE
Pressure:
    Sq. Error Mean:     0.0214799     INACCURATE
    Sq. Error Std. Dev: 0.0511252     IMPRECISE\end{lstlisting}%
\caption{Comparison of Unequal States}%
\end{subfigure}
\end{center}
% Velocity X:
%         Sq. Error Mean:                   0     ACCURATE
%         Sq. Error Std. Dev:               0     PRECISE
% Velocity Y:
%         Sq. Error Mean:                   0     ACCURATE
%         Sq. Error Std. Dev:               0     PRECISE
% Pressure:
%         Sq. Error Mean:                   0     ACCURATE
%         Sq. Error Std. Dev:               0     PRECISE
    
%     \subcaptionbox{Comparison of unequal states%
%     }[0.5\linewidth]{\begin{lstlisting}
% Velocity X:
%         Sq. Error Mean:           0.0233842     INACCURATE
%         Sq. Error Std. Dev:       0.0996487     IMPRECISE
% Velocity Y:
%         Sq. Error Mean:          0.00566354     INACCURATE
%         Sq. Error Std. Dev:       0.0139529     IMPRECISE
% Pressure:
%         Sq. Error Mean:           0.0214799     INACCURATE
%         Sq. Error Std. Dev:       0.0511252     IMPRECISE
% \end{lstlisting}}
    
    \caption{Examples outputs from the comparison tool.}
    \label{fig:example_comparisons}
\end{figure}