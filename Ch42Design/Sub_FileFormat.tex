\section{File Formats}
\label{sec:FileFormat}
To fulfil \cref{req:StoreState} two file formats have been defined to store simulation data and parameters.

\subsection{Fluid Parameters}
Parameters that are characteristic of a particular fluid or simulation type are stored in a ``Fluid Parameters'' file.
This includes the Reynolds number, the timestep safety factor, and the maximum iteration count for the Poisson solver.
They are stored in a JSON format to be human-readable, are reusable for different simulation states, and can be easily edited by the end user.
An example is shown in \cref{fig:FluidParamsExample}.

\begin{figure}[ht]
% \begin{wrapfigure}{r}{0.5\textwidth}
    \centering
    %\begin{minted}{json}
    \begin{lstlisting}[language=JSON]
{
    "Re": 150.0,
    "initial_velocity_x": 1.0,
    "initial_velocity_y": 0.0,
    "timestep_divisor": 60,
    "max_timestep_divisor": 480,
    "timestep_safety": 0.5,
    "gamma": 0.9,
    "poisson_max_iterations": 100,
    "poisson_error_threshold": 0.001,
    "poisson_omega": 1.7
}
\end{lstlisting}
% \end{minted}
    \caption{An example Fluid Parameters file.}
    \label{fig:FluidParamsExample}
% \end{wrapfigure}
\end{figure}

\subsection{Simulation State}
Data unique to an individual state such as simulation resolution, physical size, and velocity fields are stored in a binary format reused from the ACA project.
As the data is much more sensitive to individual modifications\footnote{i.e. changing a single value in the velocity field can introduce discontinuities}, it makes more sense to store this data in a binary format where it cannot be easily modified by a user.
Additionally the binary format is much smaller than any text-based format, which helps as the volume of data stored is much larger than that stored in the fluid parameters.


% \begin{figure}[ht]
    \centering
    \todomark{This will be a small diagram of the file format, which might be dropped for the Progress report if it takes too much time}
    \caption{A visual representation of the binary file format.}
    \label{fig:BinaryFileFormat}
\end{figure}
% The structure of the binary format is shown in \cref{fig:BinaryFileFormat}.
There is no magic string at the start of the file, which may be introduced in a new version.
The header consists of a pair of unsigned 32-bit integers specifying the resolution of the simulation, and a pair of 32-bit floating point numbers specifying the physical dimensions of the simulation.
From there, four sets of data for each column are stored, including the boundary padding squares:
\begin{enumerate}
    \item Horizontal Velocity $u$ (\texttt{float32})
    \item Vertical Velocity $v$ (\texttt{float32})
    \item Pressure $p$ (\texttt{float32})
    \item Cell Flags, defining which adjacent squares are boundaries (\texttt{uint8})
\end{enumerate}
This structure is somewhat unintuitive and error-prone, an example being the Cell Flags which may end up being inconsistent between adjacent cells, but for the sake of compatibility with ACA data it has been kept.
In the future it may be updated to a safer format.