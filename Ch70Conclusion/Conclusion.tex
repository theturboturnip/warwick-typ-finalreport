% !TEX root =  ../FinalReport.tex

\chapter{Conclusion}
\label{sec:Conclusion} 
% Why was this project worth doing? 

% What’s the contribution of your project? 

% What is its “technical strength”? 

% (Why was this a challenging project suitable to your degree?) Why should your project be considered a success? 


The aim of this project was to create a GPU fluid simulation and realtime in-situ visualization program, which required substantial research on fluid simulations, optimizing large parallel computation on GPUs, and various visualization methods.
Implementing the program required knowledge of C++, Vulkan, and CUDA; an in-depth understanding of the complex underlying details of each, including their handling of memory; and an effective high-level design.
%that separated the code into manageable layers.
On top of this the project was well scheduled, allowing all core features to be implemented while allowing enough time for the associated reports and presentation to be developed.
Rigorous testing was undertaken to ensure the program met the requirements, and the program passed with flying colours.
The behaviour of the simulation at scale was measured, producing interesting findings and paving the way for future work in the area.
Overall, the project has been a success.

\section{Summary}
% Novelty of in-situ realtime visualization
The full in-situ visualization is large program, at over 8.5 thousand lines of code (excluding libraries, comments, whitespace), which is impressive in its own right.
It also brings novelty as an accurate simulation/visualization that runs in real-time, rather than rendering a visualization to disk for later viewing or rendering a static simulation state.
% By using approaches from the games industry for efficient visualization, the visualization 
It uses the high-performance Vulkan rendering API, which other toolkits have been slow to adopt\footnote{VTK has a Vulkan branch at \url{https://gitlab.kitware.com/ken-martin/vtk/-/tree/vulkan/Rendering/Vulkan}, which hasn't been updated since August 2020.}.
Along with the up-and-coming Datoviz library\footnote{\url{https://github.com/datoviz/datoviz}}, it is a step forwards to bring Vulkan-level performance to the wider visualization community.

Porting the simulation to CUDA is not a new work, but it was still a significant undertaking for the researcher and required extra thought to adapt to a tightly-coupled visualization.
Gathering results at different scales demonstrated the upper limit of GPU throughput, and empirically showed the importance of cache-friendliness in GPU algorithms.
It is certainly a good starting point for future work in this space.

\section{Reflection}
While the project as a whole was successful, there are elements that could have been better executed.
Third-party libraries were used in places, most notably for the GUI, but were not used in the low-level memory management or other Vulkan code.
For Vulkan specifically, using the Vulkan Memory Allocator library \todocite{\url{https://gpuopen.com/vulkan-memory-allocator/}} would have allowed for easier memory allocation.
Other Vulkan wrappers and tools, such as the codebase developed by Sascha Willems for their Vulkan samples\todocite{https://github.com/SaschaWillems/Vulkan}, may have made certain actions less cumbersome.

A common pattern encountered when implementing new code was to design for a larger system than necessary.
For example when implementing the worker thread, a generic worker thread setup was created in case another worker thread was needed later.
Building a single worker thread would have been simpler and quicker.
In general, a lack of initial investigation on the coding side led to slight overcomplication.
However this was very minor, as most of the code developed is still used in the program.
In fact some elements, like the simulation runners, benefited from being designed as generic code first!

\section{Future Work}
% This section lists some general considerations to take into account if building a similar project, and some ideas for projects that may be interesting.
% This section lists some ideas for developing the project further, and some concepts to take into account when doing so.

On the simulation side clear areas for improvement are the cache usage/performance at scale, and the pressure inflation problem.
Both would require further investigation, but have some easy starting points.
Other parallel GPU algorithms take advantage of shared memory and locality to improve performance, which the algorithm could be adapted to support.
Investigating the non-physical pressure values fix from \cite{book:griebel1998numerical} (see \cref{ext:PressureValues}) could be the key to removing pressure inflation.
Re-implementing the Poisson residual check may reduce the number of required Poisson iterations per tick.
Different Poisson solvers could also be added to the simulation, which may be more cache-friendly.

Visualization also has many potential improvements.
A more advanced particle simulation as mentioned in \cref{sec:Evaluation:FailedReq} could be implemented, which would require more research into game industry particle simulations.
For better accessibility, the colors initially used in the visualization could also be adapted to be more colorblind friendly.

Truly parallel simulation/visualization has not been achieved, mostly due to the limitations of the researcher's hardware (\cref{sec:Design:Viz:Timing}).
As established in \cref{sec:Results:Viz:Speed} the visualization is already very fast, but for larger \& more complicated visualizations this may become a significant portion of runtime.
Using multiple GPUs, perhaps on separate systems, to implement a loosely-coupled version of this visualization could allow for a truly parallel visualization and investigation of the benefits vs. tight coupling.

All in all, this project is in a very open research area, and has a lot of potential to expand.
As massively parallel systems become more powerful and accessible, running programs like this on larger datasets will become more and more feasible, especially in industrial applications.
It will be very interesting to see what comes next.
% The researcher believes this will be watching this 
% It will be very interesting to see what comes next.
% \todomark{not-tightly-coupled, because it allows easier parallelism}
